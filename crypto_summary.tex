
\documentclass[10pt]{article} % use larger type; default would be 10pt

%% Use newline instead of indent for new paragraphs.

%%% PAGE DIMENSIONS
\usepackage{rotating}
\usepackage{times}
\usepackage{geometry} % to change the page dimensions
\geometry{a4paper} % or letterpaper (US) or a5paper or....
\geometry{margin = 2.0cm} % for example, change the margins to 2 inches all round
\geometry{top=2.0cm}
% \geometry{landscape} % set up the page for landscape
% read geometry.pdf for detailed page layout information

%\usepackage{graphicx} % support the \includegraphics command and options
\usepackage[parfill]{parskip} % Activate to begin paragraphs with an empty line rather than an indent
%%%% TIKZ Info
\usepackage[latin1]{inputenc}
\usepackage{url}
\usepackage{float}

%%% PACKAGES
\usepackage{booktabs} % for much better looking tables
\usepackage{array} % for better arrays (eg matrices) in maths
\usepackage{paralist} % very flexible & customisable lists (eg. enumerate/itemize, etc.)
\usepackage{verbatim} % adds environment for commenting out blocks of text & for better verbatim
\usepackage{subfig} % make it possible to include more than one captioned figure/table in a single float
% These packages are all incorporated in the memoir class to one degree or another...

%%% HEADERS & FOOTERS
\usepackage{fancyhdr} % This should be set AFTER setting up the page geometry
\pagestyle{fancy} % options: empty , plain , fancy
\renewcommand{\headrulewidth}{1pt} % customise the layout...
\lhead{Coursework 1}\chead{Max Robinson}\rhead{8th December 2011}
\lfoot{}\cfoot{\thepage}\rfoot{}

%%% SECTION TITLE APPEARANCE
\usepackage{sectsty}
\allsectionsfont{\sffamily\mdseries\upshape} % (See the fntguide.pdf for font help)
\paragraphfont{\sffamily\mdseries\upshape\bfseries\small} % (See the fntguide.pdf for font help)
\sectionfont{\large}
\subsectionfont{\small}
\subsubsectionfont{\small}
% (This matches ConTeXt defaults)

%%% ToC (table of contents) APPEARANCE
\usepackage[nottoc,notlof,notlot]{tocbibind} % Put the bibliography in the ToC
\usepackage[titles,subfigure]{tocloft} % Alter the style of the Table of Contents
\renewcommand{\cftsecfont}{\rmfamily\mdseries\upshape}
\renewcommand{\cftsecpagefont}{\rmfamily\mdseries\upshape} % No bold!



\title{Cryptography B: Coursework}
\author{Max Robinson \and G403 \and mr9388@bris.ac.uk}
%\date{} % Activate to display a given date or no date (if empty),
% otherwise the current date is printed

\begin{document}
\maketitle 
\section{Overview}

The paper \emph{UMAC: Fast and Secure Message Authentication}\cite{umac} discusses a
variant on Message Authentications Codes that are based the
principal of Universal Hashing. The paper then proves both the security of the
underlying algorithm with respect to current existing cryptographic
primitives.

As the title of the paper suggests, the primary advantage of using UMAC over
other schemes such as HMAC is the performance. UMAC is extremely fast with good
potential for parallelisation and therefore can be performed extremely quickly on
 multicore processes. The reason that I opted to summarise this paper is
because the performance of algorithms is not an area that has been discussed
heavily, but when it comes to real world deployment this is one of the primary
concerns (after the security of the scheme itself).

The reason that universal hashing is so fast is that the hash function used is
not designed to be secure. This is counter intuitive for a security scheme but
the hash function is used to compress the incoming message so that a secure,
but slower, scheme can be applied to this digest to provide the final output.
This summary looks at some of the keys aspects of the UMAC approach and then
concludes with an evaluation of the approach.

\section{Universal hashing}

The paper relies of the principal of universal hashing in order to compress the
message before signing. With universal hashing instead of considering a single
hash function (keyed or unkeyed), we have a large family of possible hash
functions and one of these is then chosen (either at random or by some mapping
of keys) and used by the algorithm.

A set of hash functions mapping from domain D to range R is said to be
universal if Equation \ref{eqn:universal-hashing} holds when the function h is drawn
randomly from H. Informally, the collision probability for a hash function is
exactly the probability of two random strings colliding ($\frac{1}{|R|}$).

\begin{equation}
Pr_{h \in H}[h(x) = h(y)] = \frac{1}{|R|}\ \forall x \neq y \in D
\label{eqn:universal-hashing}
\end{equation}

In practice, and in this paper, a weaker definition is instead used,
$\epsilon$-almost universal, defined for the set set of h by Equation
\ref{eqn:almost-universal}. Here
the collision probability is bounded by some small fixed constant
($\frac{1}{|R|} < \epsilon < 1$). This concept of $\epsilon$-AU is essential
for providing the security bounds for the algorithm (if we have a high
probability of a collision then our MAC will not be secure).

\begin{equation}
Pr_{h \in H}[h(x) = h(y)] = \epsilon\ \forall x \neq y \in D
\label{eqn:almost-universal}
\end{equation}

%An even stronger notion is also introduced, pairwise independence, which is
%shown in Equation \ref{eqn:strongly-universal}. In other words, all hash
%functions in the set operate completely indepedently of the message being
%hashed. A set of functions that meets this requirement is said to be strongly
%universal, and the corresponding weaker definition of $\epsilon$-almost
%strongly universal for a probability ($\frac{\epsilon}{|R|})$.
%
%\begin{equation}
%Pr_{h \in H} [h(x) = a, h(y) = b] = \frac{1}{|R|^2} \forall x \neq y \in D,
%\forall a,b \in R
%\label{eqn:strongly-universal}
%\end{equation}

 %TODO Talk about how fast UH means a strong encryption can be chosen for
 %encrypt step.
\section{MAC construction using universal hashing}
The Wegman-Carter paradigm for a universal hash based MAC is as follows:

Assumed two parties have agreed a hash function(h) from the family of hash
functions(H)  and have shared a set of one time pads $d_1,d_2,\cdots,d_n$. To
authenticate the message $m_i$ we compute $h(m_i)+d_i$.

If $h$ is drawn from an $\epsilon$-almost-strongly universal hash family, the
probability of a computationally unbounded adversary breaking the encryption
is at most $\epsilon$. In practice, the one time pad is substituted for a
pseudo random function (PRF) which weakens the scheme to the probability that
the PRF can be broken plus $\epsilon$.

\section{Existing hash families}

\paragraph{MMH}

The MMH hash function was first defined in \emph{MMH: Software message
authentication in the Gbit/second rates}\cite{mmh} and defines a hash function
in finite fields over a prime $p$, and any integer $k > 0$ such that:
\begin{equation}
MMH = \{g_x : Z^k_p \rightarrow Z_p | x \in Z^k_p\}
\end{equation}
\begin{equation} \label{eqn:mmh}
g_x(m) = m \cdot x\ mod\ p = \displaystyle\sum^k_{i=1} m_i \cdot x_i\ mod\ p
\end{equation}

Given a $k$-block message where each $m_i$ is a member of the finite field
$Z_p$. To compute the hash we multiply each block by its corresponding key and
return the sum mod $p$. The function used for the UMAC algorithm is a variation
on this MMH hash family called NH.

\paragraph{NH}

The NH hash family is defined in Equation \ref{eqn:nh}. We specify a word size ($w$), usually 16 or 32 bits to exploit fast
modulo operations in hardware. We then represent the message M as an even
number of $w$-bit blocks. ($M = (m_1,\cdots,m_l)$ where $m_i \in
{0,\cdots,2^w-1}$). We also have a key,
$k$ with length $n
\geq l$. In essence, for each pair of blocks in M, both blocks are added to
the corresponding key (the answers are returned modulo the maximum word value)
then multiplied together. The result of this is then summed for each pair of
blocks and the result is returned modulo the maximum word size squared. This
means each hash is at most $2w$ bits long. 

\begin{equation} \label{eqn:nh}
NH_k(m) = \left( \sum^{l/2}_{i=1} \left( \left( \left( m_{2i-1} + k_{2i-1}
\right)\ mod\ 2^w
\right) \cdot \left( \left( m_{2i} + k_{2i} \right)\ mod\ 2^w \right) \right)
\right) mod\ 2^{2w}
\end{equation}
\paragraph{NH Collision Bound}

The paper then goes on to prove that the collision bound for NH is $2^{-w}$.
This is accomplished by showing that, assuming the messages differ in one
block, there is at most \emph{one} key from the set of all keys that causes a
collision.

\paragraph{Toeplitz extension}

The paper also discusses methods to reduce this collision probability further
(from $2^{-w}$ to $2^{-2w}$ for example). This was accomplished using the
Toeplitz-extension ( a more efficient method that extending the word size or
concatenating multiple hashes).

The principal is that the key is extended so that you can take a series of
overlapping keys, for example $K_1 = k_1,k_2,\cdots,k_n$ and $K_2 =
k_3,k_4,\cdots,k_{n+2}$. This means that the key need only be extended by a
number of blocks equal to twice the number of additional hashes you will be using
(Each one decreases by collision probability by $2^{-w}$).

Interesting, despite the fact that keys are related, the paper proves that the
collision probability still drops to its desired level. Here it is proven that
the probability that one of these key shifts collides given that all the other
key shifts collide is at most $2^-w$, i.e. the fact a collision has occured
does not increase the likelihood of future collisions. Intuitively, this is
shown by fixing all bits of the key except for the new bits (the ones that have
been shifted into use). Then using the same technique that we used to prove the
collision bound of NH, we show there is at most one key from these new key
blocks that will cause a collision.

\paragraph{Padding}

A significant drawback for the NH scheme is that it can only operation on
strings that we an even number of $w$-bit words long. It is therefore necessary
to incorporate a method pf padding such that it can be used for any length
strings. Fortunately this modification is not complicated and takes the
following form:
\begin{enumerate}
\item Divide the message into a sequence of blocks Msg =
$Msg_1||Msg_2||\cdots||Msg_t$ such that all blocks have length in bits =
blocksize($n$) except for $Msg_t$ which is shorter.
\item Assign Len = $|Msg|\ mod\ a$ (the length of the final block)
\item Pad $Msg_t$ with zeros until its $|Msg_t| = 0\ mod 2w$
\item Return $h(Msg_1)||\cdots||h(Msg_t)||Len$
\end{enumerate}

The length is appended to the end of the message to ensure that messages that
are not equal length will not collide because the \emph{Len} will always be different.

\section{Converting NH to a MAC}

NH by itself is not secure, due to its construction using arithmetic it is
trivial to find K. This can be done by the following:
\begin{enumerate}
\item Set $M = m_1||m_2$ for some random $m_1,m_2$
\item Get $H(M) = m_1m_2 + m_1k_2 + m_2k_1 + k_1k_2$
\item Set $M' = m_1+1||m_2$
\item Get $H(M') = (m_1+1)m_2 + (m_1+1)k_2 + m_2k_1 + k_1k_2$
\item Compute $C = H(M) - H(M') = m_2 + k_2$ from which $m_2$ is known
\item $k_2 = C - m_2$ or $k_2 = C - m_2 + 2^w$ (if $k_2 + m_2 > 2^w)$
\end{enumerate}

This process can be repeated to recover the entire key. (Although $k_2$ is
sufficient to produce forgeries using $m_i' = m_i+1$ by adding $m_2+k_2$ to the
hash of the message containing $m_i$).

It is therefore necessary to do something to the output so that this attack is
not possible. This is accomplished by performing an additional step on the
output of NH using a pseudo random function (PRF). This step involves passing
the output and a nonce (some random string) through the PRF and using this as
the tag. The scheme is then given below:

Key Generation: Select a pseudo random function ($f$) at random from a family of
pseudo random functions ($F$). Select a hash function ($h$) at random from our family
of NH hash functions ($H$)\\
TAG : Given message (M) and a Nonce (randomly generated string, can be a simple
counter) output $f(h(m)||Nonce)$

The PRF stops the output of our MAC scheme from being linearly related to our
message which is essential for a secure scheme. The security of this step is
linearly related to the security of our scheme as a whole. The paper proves
that the security of our MAC scheme is $\epsilon + \delta$ where $\epsilon$ is
the collision probability of our hash and $\delta$ is the probability that an
adversary can break the PRF (distinguish between the output of the function and
a truly random function).

The nonce is necessary to prevent a birthday attack against the system.
Without this then after making $2^{-\frac{w}{2}}$ queries we would have a 50\%
chance of finding a collision and therefore a forgery for our scheme. However,
with the nonce these attacks are not possible because we cannot reuse the same
nonce under the same key (verifier will reject the message).

\subsection{UMAC}

The UMAC protocol requires a pseudo-random generation (PRG), a shared secret
key and the ability to generate a nonce (can be a simple counter i.e. the
sender must be stateful). The paper details a special case of UMAC, shown in figure
(something here) and a detailed explaining with reference to this example is
given below.

\paragraph{Key generation}

The construction of the NH family requires a key that is at least as long as
the message, since sharing a key like this would be impractical for large
messages, instead the PRG is used to generate this key to get: $K =
K_1K_2\cdots K_{1024}$ for $K_i \in \{0 \cdots 1\}^{32}$.

An additional key $A$ is also generated for $|A|=512$ to be used to key the
SHA1 used in Step %TODO

This key generation is likely expensive and therefore the two keys are only
generated at the start of the session then reused.

\paragraph{Hashing}
The generating of the Tag takes place is two steps, hashing and tagging. The
hash step works as follows
\begin{itemize}
\item Let Len be $|Msg|$ mod 4096*
\item Pad $Msg$ with 0 bits such that $|Msg|\ mod\ 8* \equiv 0$
\item Let $Msg=Msg_1||Msg_2||\cdots||Msg_t$ where $|Msg_i|$ is 1024 words (except
for $Msg_t$)
\item $HM = NH_K(Msg_1)||NH_K(Msg_2)||\cdots||NH_K(Msg_t)||Len$
\end{itemize}
*Although not explicit mentioned in the paper, we assume the the 4096 and 8
here refer to the length in bytes rather than in bits. This then means that Len
is the length of the final message in bytes. The padding will then also ensure
there are an even number of 32 bit words when doing the final $NK_K$ for block
$MSG_t$.

The NH function is used in a way that is similar to that of a block cipher.
With each ``block'' of HM independent of the blocks on either side. 


\section{Closing thoughts}

The UMAC scheme is interesting in that the hash scheme exists more as a means
of achieving secure compression instead of computing the tag itself. This
compression allows a secure function to be used on the output to generate the
tag efficiently (since the output of the NH family is small).

The collision probability of the NH family is not based upon any cryptographic
primitives which means that the security of the scheme is not reliant on any
assumptions that may later be proved false. In practise, the PRF used will be
based on certain assumptions which, if invalidated, may necessitate the use of
a different function during implementation.

The flexibility of the scheme is particularly interesting, the variability of
the word size and the possibility of the extensions using Toeplitz construction
allows the security and performance of the scheme to be traded off in a way
that is easy to understand. This increases the usability of the scheme by
catering for different security needs, the paper itself also implementations of
some of these variations which aides deployment. In the future works they discuss the opportunity
for a single MAC that can be verified at increasing strengths. This is
particularly interesting, however, lack of further research into this area
suggests that it may not have been as simple as the paper implied.

However, despite this significant performance advantages there are a few major
drawbacks to the system. The first is the advantage that the scheme uses a
different scheme to actually produce the final output is advantageous in that
it allows flexibility but it adds an additional layer of complexity to
implementation. The UMAC scheme does not only need to be implemented but so does
whatever implementation is used for the PRF (and the PRG for the key
generation). Although the paper provided specifications for different security
thresholds which do simplify this process somewhat.

The first is that the size of the key needed is
significant, at least as long as the block size for each message plus a key
required for the PRF. The authors
dismiss this claiming that this key generation is only required once, however
that is only true for extended sessions whereby information is exchanged using
the same key. In practise, for shorter sessions the cost of generation will
likely offset any performance gains that may be achieved.

Secondly, the scheme is stateful, requiring that a nonce is tracked by the
server which is never reused. This complicates the process and means that guard
must be in place to prevent this nonce from being attacked through side channels.

Finally, the question becomes whether the advancements that have been made in
processing reduce the cost of performing a MAC to the point that increases in
the speed of this process (even those as significant as UMCA) become
unnecessary.

Ultimately, given the lack of widespread adoption 14 years on from the
publication of the paper, we must conclude that the added complexity of the
implementation is not justified by the increase in speed.
\bibliographystyle{plain}
\bibliography{crypto_summary}
\end{document}
