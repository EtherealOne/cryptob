
\section{Overview}

The paper \emph{UMAC: Fast and Secure Message Authentication} discusses a
variants on Message Authentications Codes that are based primarily on the
principal of Universal Hashing. The paper then proves both the security of the
underlying algorithm, with respect to current existing cryptographic
primatives.

As the title of the paper suggests, the primary advantage of using UMAC over
other schemes such as HMAC is the performance. UMAC is extremely fast and is
inherently parallellisable and therefore can be performed extremely quickly on
 multicore processes. The reason that I opted to summarise this paper is
 because the performance of algorithms is not an area that has been discussed
 heavily, but when it comes to real world deployment this is, easily, the
 second most important concern (after the security of the scheme itself).

 \section{Primitives}

 \subsection{Universal hashing}

 The paper relies of the principal of universal hashing in order to compress
 the message before signing. Universal hashing is a different approach to
 other hashing methods in that instead of considering a single hash function
 (keyed or unkeyed), we have a large familiy of possible hash functions and
 one of these is then chosen (either at random or by some mapping of keys) and
 used by the algorithm.

 Before we continue with universal hashing there is one key notion that we
 must be familiar with, the concept of an $\eta -universal$ hash family.
 Informally, a family is said to be $\eta -universal$ if for any pair fo
 distinct message, the probability that they collide is at most $\eta$ if the
 hash function used to hash each message is chosen randomly from the set of
 hash functions.
 %TODO Formal definition / proof here?

 %TODO Explain why we picked UH? Do it here?

 %TODO Look up the combinatorial property of universal hashing

 %TODO Talk about how fast UH means a strong encryption can be chosen for
 %encrypt step.

 In the paper, the universal hashing family of choice is known as NH hashing.
 It is a simplification of the NMH and MMH families
 %TODO Lookup these and talk about them
 The principal of NH is described in \ref{equ:nh}
\begin{algorithm}
$\left( NH_k(m) = \sum^{t/2}_{i=1} \left( \left( \left( m_{2i-1} + k_{2i-1} \right) mod 2^w
\right) \cdot \left( \left( m_{2i} + k_{2i} \right) od 2^w \right) \right)
\right) mod 2^{2w}$
\label{eqn:nh}
\end{algorithm}

The algorithm defines a word size ($w$), usually 16 or 32 bits to exploit fast
modulo operations in hardware. We then represent the message M as an even
number of $w-bit$ blocks.

$M = (m_1,\cdots,m_l)\ where\ m_i \in {0,\cdots,2^w-1}$


